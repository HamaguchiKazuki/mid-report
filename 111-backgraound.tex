\section{背景}
\par 現在、ボイスチェンジャーと呼ばれているものが世の中に多く存在している。声質変換 (Voice Conversion: VC) とはボイスチェンジャーの一種であり、まるである人物の入力音声を特定の人物が話したかのように変換して出力する手法である。特に個人やバーチャルユーチューバーのリアルタイムで行われる動画配信において、声質変換が利用されている。声質変換の1つに、入力音声を一度テキストに変換させ、変換させたテキストから合成音声を出力する手法がある。ここで、入力音声を一度テキストに変換する段階を音声認識とする。また、テキストから合成音声を出力する段階をテキストベース音声合成 (Text-to-speech: TTS) とする。この音声認識とテキストベース音声合成を組み合わせた手法にはいくつかの問題がある。
\bunseki{}

