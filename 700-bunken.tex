\begin{thebibliography}{9}
  \bibitem{1}  T. Merritt Z. Wu O. Watts, G. E. Henter and S. King, “From HMMs to DNNs: where do the improvements come from?,” in Proc. ICASSP, Shanghai, China, Mar. 2016, pp. 5505–5509.
  \bibitem{2} S. Fan, Y. Qian, and F. Soong,
“TTS synthesis with bidirectional LSTM based recurrent neural networks,”
in Proc. INTERSPEECH, Max Atria, Singapore, Sep. 2014, pp. 1964–1968.
  \bibitem{3} H. Zen and H. Sak, “Unidirectional long short-term memory recurrent neural network with recurrent output layer for low-latency speech synthesis,” in Proc. ICASSP, Brisbane, Australia, Apr. 2015, pp. 4470–4474.
  \bibitem{4}  S. Fan, Y. Qian, and F. Soong, “TTS synthesis with bidirectional LSTM based recurrent neural networks,” in Proc. INTERSPEECH, Max Atria, Singapore, Sep. 2014, pp. 1964–1968.
  \bibitem{5} L. Sun, K. Li, H. Wang, S. Kang, and H. Meng, “Phonetic posteriorgrams for many-to-one voice conversion without parallel data training,” Multimedia and Expo (ICME), 2016 IEEE International Conference on, 2016. pp. 1-6
  \bibitem{6} Hiho (2018)「ディープラーニングの力で結月ゆかりの声になってみた」,
 < https://blog.hiroshiba.jp/became-yuduki-yukari-with-deep-learning-power/ >2018年7月18日アクセス
  \bibitem{7} Yoshikazu Oota(2018)「ディープラーニングの力で結月ゆかりの声になる」ための基礎知識とコマンド操作」, < https://qiita.com/atticatticattic/items/575d71dab4ee716e4969 >2018年7月18日アクセス
  \bibitem{8} さんこ(2017)「自分の声を水瀬いのりさんの声にする他対1声質変換」,< http://sesenosannko.hatenablog.com/entry/inori\_vc1 >2018年7月20日アクセス
  
  %本文では使っていない
  \bibitem{9} 斎藤康毅(2016)「ゼロから作るディープラーニング」, O’Reilly Japan
  
\end{thebibliography}
