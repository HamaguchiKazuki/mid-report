% プロジェクト学習中間報告書書式テンプレート ver.1.0 (iso-2022-jp)

% 両面印刷する場合は `openany' を削除する
\documentclass[openany,11pt,papersize]{jsbook}

% 報告書提出用スタイルファイル
%\usepackage[final]{funpro}%最終報告書
\usepackage[middle]{funpro}%中間報告書
% 画像ファイル (EPS,EPDF,PNG) を読み込むために
%\usepackage[dvipdfmx]{graphicx,color}

% 年度の指定
\thisYear{2018}

% プロジェクト名
\jProjectName{AIするディープラーニング}

% [簡易版のプロジェクト名]{正式なプロジェクト名}
% 欧文のプロジェクト名が極端に長い(2行を超える)場合は,短い記述を
% 任意引数として渡す.
%\eProjectName[Making Delicious curry]{How to make delicious curry of Hakodate}
\eProjectName{AI Love Deep Learning Project}


% <プロジェクト番号>-<グループ名>
\ProjectNumber{06}

\begin{document}

%
% 表紙
%\maketitle
%

%\newpage
\pagestyle{empty}

%表紙
\begin{center}
{\Large 公立はこだて未来大学 2018年度 システム情報科学実習\\[-1mm]グループ報告書\\[2mm] Future University Hakodate 2018 System Information Science Practice\\ [-1mm]Group Report\\[10mm]}
{\large プロジェクト名\\[-1mm]AIするディープラーニングプロジェクト\\[1mm]Project Name\\[-1mm]Future Mobile Phone Project}\\[10mm]
プロジェクト番号 / Project No\\
06\\[10mm]
{\normalsize 大学名 / University Name\\[-1mm]公立はこだて未来大学/ Future University Hakodate}\\[5mm]
{\normalsize プロジェクトリーダ/Project Leader}\\[-1mm]
\ \ \ 1016163 濱口 和希\ \ \ /Kazuki Hamaguchi\\[10mm]
{\normalsize プロジェクトリーダ/Group Leader}\\[-1mm]
\ \ \ 1016163 濱口 和希\ \ \ /Kazuki Hamaguchi\\[10mm]
\begin{center}
プロジェクトメンバ/Project Member\\
\leftskip=14zw
1016163 濱口 和希\ \ \ \ \ \ \ /Kazuki Hamaguchi\\
1016065 白鳥 孝幸\ \ \ \ \ \ \ /Takayuki Shiratori\\
1016132 山田 大貴\ \ \ \ \ \ \ /Daiki Yamada\\
1015205 齋藤 匠\ \ \ \ \ \ \ \ /Takumi Saitou \\
[10mm]
\end{center}
指導教員/Advisor\\
竹之内 高志/Takashi Takenouchi\\
香取 勇一/Yuuichi Katori\\
寺沢 憲吾/Kengo Terasawa\\
片桐 恭弘/Yasuhiro Katagiri\\
冨永 敦子/Atsuko Tominaga\\


提出日/Date of Submission\\
2018年7月20日/July 20,2018\\
\end{center}


%前付け
\frontmatter

% 和文概要
\begin{jabstract}

 声質変換 (Voice Conversion: VC) とはボイスチェンジャーの一種であり、まるである人物の音声を特定の人物が話したかのように変換する手法である。既存の手法は、入力音声を一度テキストに変換させ、変換させたテキストから合成音声を出力している。しかし、この手法では問題点が3つある。1つ目は、複数のツールを使用しているため、入力から出力までが遅い点である。2つ目は、音声をテキストに変換する際、誤認識が多い点である。3つ目は、変換させたテキストから合成音声を出力する際、変換先が限られてしまう点である。そこで、本グループではこれらの問題点をディープラーニングを駆使して改善を試みる。




\end{jabstract}


%英語の概要
\begin{eabstract}
Voice conversion is a kind of voice changer and it is a method of converting the voice as if a specific person spoke. The previous method converts input speech to text once and outputs synthesized speech from the converted text. But this method has three problems. First, because using multiple tools, it is slow from input to output. Second, there are many misrecognition when converting voice to text. Third, when synthesized speech is output from converted text, conversion destination is limited. Therefore, in our group we try to improve these problems using deep neural network.
\\[5mm]

\end{eabstract}


%目次
\tableofcontents% 目次

\mainmatter% 本文のはじまり

%
% 第一章 はじめに
%
\chapter{はじめに}

%背景
\section{背景}
\par 現在、ボイスチェンジャーと呼ばれているものが世の中に多く存在している。声質変換 (Voice Conversion: VC) とはボイスチェンジャーの一種であり、まるである人物の入力音声を特定の人物が話したかのように変換して出力する手法である。特に個人やバーチャルユーチューバーのリアルタイムで行われる動画配信において、声質変換が利用されている。声質変換の1つに、入力音声を一度テキストに変換させ、変換させたテキストから合成音声を出力する手法がある。ここで、入力音声を一度テキストに変換する段階を音声認識とする。また、テキストから合成音声を出力する段階をテキストベース音声合成 (Text-to-speech: TTS) とする。この音声認識とテキストベース音声合成を組み合わせた手法にはいくつかの問題がある。
\bunseki{}



%問題点
\section{現状における問題点}
\par 先ほど述べた、音声認識とテキストベース音声合成を組み合わせた手法は、入力から出力までが遅くなる。また、音声をテキストに変換する際、誤認識が発生する。さらに、変換させたテキストから合成音声を出力する際、変換先が限られる。
\bunseki{(未来大)}


%目的
\section{目的}
\par 本グループの目的は、音声認識とテキストベース音声合成を組み合わせた手法よりも高性能な変換ツールを開発することである。その後、開発したツールとテキストベース声質変換ツールをグループ内で検証し、性能評価を行う。
\bunseki{}


%
% 第二章 プロジェクト学習の概要
%
\chapter{グループ課題設定までのプロセス}

%2.1 合同プロジェクト
\input{210-unitproject.tex}

%2.2 課題の設定
\section{期間ごとの課題設定}
\subsection{前期の課題設定}
\par
さしすせそ

\subsection{後期の課題設定}
\par
後期に執筆する。


%2.3 課題の割り当て
\section{担当の割り当て}
\subsection{前期担当の割り当て}

\subsection{後期担当の割り当て}


%2.4 組織形態
\section{担当分担課題の評価}



%
% 第三章 活動内容
%
\chapter{活動内容}

%
% Cool Japanimation
%

%3.1 前期の活動内容
\section{前期の活動内容}
\subsection{従来の声質変換手法の調査}

\subsection{音声の作成・収集}

\subsection{先行事例の追実験}

\subsection{中間発表の振り返り}


%3.2 前期の個人活動
\section{前期の個人活動}
\subsection{濱口和希}
\par

\subsection{白鳥孝幸}
\par
前期では、webサイトにて声質変換に関連する文献を調査・検討した。使用する手法についての考察を、ネットワーク班の中で考察した。また、PCにUbuntuを導入し、開発環境を整えた。Pythonや機械学習に関するフレームワーク、データ解析についての学習を自主的に行った。

\subsection{山田大貴}

\subsection{齋藤匠}

%
% 第四章 開発手法について
%
\chapter{開発手法の説明}

%4.1.0 開発プロセスの概要
\input{410-development.tex}

%
% 第五章 今後の課題
%
\chapter{今後の課題}

%5.1.0 活動内容 CoolJapanimation
\section{前期終了時の課題}

\section{後期終了時の課題}


%
% 第六章 まとめ
%
\chapter{まとめ}

%6.1.0 Cool Japanimation


%6.2.0 Rhyth/Walk
\section{年間を通しての成果}

\section{前期の成果}

\section{後期の成果}


%\backmatter

% 参考文献
\begin{thebibliography}{9}
  \bibitem{1}  T. Merritt Z. Wu O. Watts, G. E. Henter and S. King, “From HMMs to DNNs: where do the improvements come from?,” in Proc. ICASSP, Shanghai, China, Mar. 2016, pp. 5505–5509.
  \bibitem{2} S. Fan, Y. Qian, and F. Soong,
“TTS synthesis with bidirectional LSTM based recurrent neural networks,”
in Proc. INTERSPEECH, Max Atria, Singapore, Sep. 2014, pp. 1964–1968.
  \bibitem{3} H. Zen and H. Sak, “Unidirectional long short-term memory recurrent neural network with recurrent output layer for low-latency speech synthesis,” in Proc. ICASSP, Brisbane, Australia, Apr. 2015, pp. 4470–4474.
  \bibitem{4}  S. Fan, Y. Qian, and F. Soong, “TTS synthesis with bidirectional LSTM based recurrent neural networks,” in Proc. INTERSPEECH, Max Atria, Singapore, Sep. 2014, pp. 1964–1968.
  \bibitem{5} L. Sun, K. Li, H. Wang, S. Kang, and H. Meng, “Phonetic posteriorgrams for many-to-one voice conversion without parallel data training,” Multimedia and Expo (ICME), 2016 IEEE International Conference on, 2016. pp. 1-6
  \bibitem{6} Hiho (2018)「ディープラーニングの力で結月ゆかりの声になってみた」,
 < https://blog.hiroshiba.jp/became-yuduki-yukari-with-deep-learning-power/ >2018年7月18日アクセス
  \bibitem{7} Yoshikazu Oota(2018)「ディープラーニングの力で結月ゆかりの声になる」ための基礎知識とコマンド操作」, < https://qiita.com/atticatticattic/items/575d71dab4ee716e4969 >2018年7月18日アクセス
  \bibitem{8} さんこ(2017)「自分の声を水瀬いのりさんの声にする他対1声質変換」,< http://sesenosannko.hatenablog.com/entry/inori\_vc1 >2018年7月20日アクセス
  
  %本文では使っていない
  \bibitem{9} 斎藤康毅(2016)「ゼロから作るディープラーニング」, O’Reilly Japan
  
\end{thebibliography}

% \begin{thebibliography}{9}
% \end{thebibliography}

\end{document}
